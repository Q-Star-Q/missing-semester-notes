% Options for packages loaded elsewhere
\PassOptionsToPackage{unicode}{hyperref}
\PassOptionsToPackage{hyphens}{url}
%
\documentclass[
]{article}
\usepackage{amsmath,amssymb}
\usepackage{lmodern}
\usepackage{iftex}
\ifPDFTeX
  \usepackage[T1]{fontenc}
  \usepackage[utf8]{inputenc}
  \usepackage{textcomp} % provide euro and other symbols
\else % if luatex or xetex
  \usepackage{unicode-math}
  \defaultfontfeatures{Scale=MatchLowercase}
  \defaultfontfeatures[\rmfamily]{Ligatures=TeX,Scale=1}
\fi
% Use upquote if available, for straight quotes in verbatim environments
\IfFileExists{upquote.sty}{\usepackage{upquote}}{}
\IfFileExists{microtype.sty}{% use microtype if available
  \usepackage[]{microtype}
  \UseMicrotypeSet[protrusion]{basicmath} % disable protrusion for tt fonts
}{}
\makeatletter
\@ifundefined{KOMAClassName}{% if non-KOMA class
  \IfFileExists{parskip.sty}{%
    \usepackage{parskip}
  }{% else
    \setlength{\parindent}{0pt}
    \setlength{\parskip}{6pt plus 2pt minus 1pt}}
}{% if KOMA class
  \KOMAoptions{parskip=half}}
\makeatother
\usepackage{xcolor}
\IfFileExists{xurl.sty}{\usepackage{xurl}}{} % add URL line breaks if available
\IfFileExists{bookmark.sty}{\usepackage{bookmark}}{\usepackage{hyperref}}
\hypersetup{
  hidelinks,
  pdfcreator={LaTeX via pandoc}}
\urlstyle{same} % disable monospaced font for URLs
\setlength{\emergencystretch}{3em} % prevent overfull lines
\providecommand{\tightlist}{%
  \setlength{\itemsep}{0pt}\setlength{\parskip}{0pt}}
\setcounter{secnumdepth}{-\maxdimen} % remove section numbering
\ifLuaTeX
  \usepackage{selnolig}  % disable illegal ligatures
\fi

\author{}
\date{}

\begin{document}

\hypertarget{shell-prompt}{%
\subsubsection{Shell Prompt}\label{shell-prompt}}

\textbf{可视化界面(这里说是Visual Interface)实际上一般说GUI(Graphiscal
User Interface)}\\
Shell: 命令行界面(CLI,Command Line Interface)\\
Windows: Powershell,Windows7以前没有pwsh。\\
Linux,OSX:一般使用Bash(Bourne Again Shell)。\\
Terminal外观:单行\\
例如

\begin{verbatim}
[jon@xpanse missing-semseter]$
\end{verbatim}

'\$'左边这一串叫做\textbf{命令行提示符(Shell
Prompt)},通常包含用户名(username),机器名称(name of
machine),路径(当前的PATH)。\\
'\$'右边闪烁的光标是在请求你的输入。\\
命令通常是带着参数(argument)的执行程序。带着参数去执行,可以修改程序的行为。\textbf{参数:紧随程序名称后面,用空格分隔的东西。}

\hypertarget{shellux600eux4e48ux77e5ux9053ux8fd9ux4e9bux7a0bux5e8fux8981ux505aux4ec0ux4e48}{%
\subparagraph{Shell怎么知道这些程序要做什么?}\label{shellux600eux4e48ux77e5ux9053ux8fd9ux4e9bux7a0bux5e8fux8981ux505aux4ec0ux4e48}}

计算机操作系统通常拥有自己的\textbf{内置程序(Built-in)},通常内嵌了终端程序,或者是Windows
Explorer,浏览器这类内嵌了围绕终端工作的程序。\\
这些程序位于File
System(文件系统),让Shell有办法知道程序放在目录的什么地方。

\hypertarget{date---ux663eux793aux65e5ux671f}{%
\paragraph{date - 显示日期}\label{date---ux663eux793aux65e5ux671f}}

举例

\begin{verbatim}
$date
Wed Aug 11 10:28:58 CST 2021
\end{verbatim}

\hypertarget{echo---ux8fd4ux56deux7a7aux683cux540eux9762ux5e26ux7684ux53c2ux6570arugument}{%
\paragraph{echo -
返回空格后面带的参数(arugument)}\label{echo---ux8fd4ux56deux7a7aux683cux540eux9762ux5e26ux7684ux53c2ux6570arugument}}

\begin{verbatim}
$echo hello
hello
\end{verbatim}

或者

\begin{verbatim}
$echo "Hello World"
Hello World
\end{verbatim}

单转义字符(Escape character)

\begin{verbatim}
$echo Hello\ World
Hello World
\end{verbatim}

\hypertarget{environment-variable}{%
\paragraph{Environment Variable}\label{environment-variable}}

环境变量,类似于程序设计语言的变量(variable),也就是说,Shell或者说Bash本身就是一种程序设计语言。\\
它们的Prompt能带参数运行程序,还能写出while循环,for循环,条件循环语句或者是定义函数(详细的将在Shell
Scripting说明)。

\hypertarget{ux4ec0ux4e48ux662fux73afux5883ux53d8ux91cf}{%
\subparagraph{什么是环境变量?}\label{ux4ec0ux4e48ux662fux73afux5883ux53d8ux91cf}}

环境变量是Shell本来就设定好的东西,无论何时去启动Shell都不需要你去重新设置。这类设置好的东西,例如
home目录,username用户名(路径PATH是做这类工作的)。\\
目录:这里说的是Shell寻找程序时所查找的目录,计算机会寻找与你输入指令名字相同的一个程序或者文件。例如你输入\texttt{date}或者\texttt{echo},计算机遍历目录,直到找到它们。\\
如果你想知道程序是在哪个目录运行的,可以这样做

\begin{verbatim}
$which
# 如果是echo,就会返回echo所在的目录
/usr/bin/echo
\end{verbatim}

说到目录,这里提及一下PATH。\\
\textbf{PATH:描述你的计算机文件位置的东西。例如echo的示例,你会发现在/usr前面有一个根目录(root),这是因为PATH的起点是根目录,它是整个文件系统的最顶层。}\\
这里需要说明:\\
Linux,OSX的目录都是绝对路径,也就是说,所有的东西都在一个命名空间(namespace)里。\\
Windows比较特殊,它的每个分区都有一个根目录,比如\texttt{C:\textbackslash{}},每个驱动器下,都是独立的。这是Windows每个盘符都有一套独立的文件系统的层次结构。

\hypertarget{print-working-directorypwd}{%
\paragraph{print working
directory(pwd)}\label{print-working-directorypwd}}

打印当前工作目录(print working directory)

\begin{verbatim}
$pwd
/home/Shaymin
\end{verbatim}

\hypertarget{change-directorycd}{%
\paragraph{change directory(cd)}\label{change-directorycd}}

改变工作目录(change directory)

\begin{verbatim}
# 改变前目录/home/Shaymin
$cd /home
$pwd
/home
\end{verbatim}

\hypertarget{--}{%
\paragraph{'.' \& '..'}\label{--}}

特殊字符单引号(.)和双引号(..)。

\begin{verbatim}
# '.'表示当前目录,'..'表示上一层目录。
# 当前目录/home
$cd ../
$pwd
/
# 返回根目录了。这里表示的都是相对路径。
$cd -
/home
# 如果输入上面指令,将会返回改变前目录。
$cd ~
/home
# 如果这样做,一定会返回/home目录。
\end{verbatim}

这里需要说明,一般用户给出程序名称,Shell会用PATH去查找位置,默认在当前目录。

\hypertarget{listls}{%
\paragraph{\texorpdfstring{\textbf{list(ls)}}{list(ls)}}\label{listls}}

显示当前目录的文件

\begin{verbatim}
$ls
picture video download ...
# 输入文件或者文件夹,名称之间会用空格分隔。

# 如果想快速显示上一层目录的文件,键入以下任一指令
$cd ../
$ls
# or
$ls ../

# 如果这样做会显示当前目录的权限。
$ls -l
# 意思是 use a long listing format.
# 示例输出
drwxrwxrwx 1 shaymin shaymin 4096 Aug 11 08:35 missing-semester-notes
\end{verbatim}

说明

d - directory\\
rwxrwxrwx(权限):由后面的9个字母分成三组组成,它们分别代表了三个不同的用户组。\\
1.计算机所有者\\
2.拥有文件的用户组\\
3.非所有者的其他人\\
r - read\\
w - write\\
x - excute\\
\textbf{注意:只有文件w权限,没有它的整个路径目录的w权限,是不能够删除的。除了目录。}

\hypertarget{rename-filemv}{%
\paragraph{rename file(mv)}\label{rename-filemv}}

通过改变文件的所在目录和名称,可以进行重命名或者移动文件位置的操作。

\begin{verbatim}
$ls
test.txt
$mv /home/Shaymin/test.txt /home/Shaymin/hello.txt
# 在目录home/Shaymin/下的test.txt文本文件被重命名为hello.txt
$ls
hello.txt
\end{verbatim}

\hypertarget{copy-filecp}{%
\paragraph{copy file(cp)}\label{copy-filecp}}

复制文件,格式 {[}复制源路径{]} {[}目标路径{]}

\begin{verbatim}
$cp /home/Shaymin/hello.txt /home/hello.txt
# 此时文件复制了一份到/home 目录下。
\end{verbatim}

\hypertarget{remove-filerm}{%
\paragraph{remove file(rm)}\label{remove-filerm}}

移除文件(对目录无效)除非添加参数。

\begin{verbatim}
[shaymin@ubuntu2004 /home/Shaymin]$rm hello.txt
# 不会返回信息,但文件hello.txt已被删除。
\end{verbatim}

\hypertarget{make-directorymkdir}{%
\paragraph{make directory(mkdir)}\label{make-directorymkdir}}

创建一个文件夹目录

\begin{verbatim}
$ls
picture video download
$mkdir "my photos"
# 特别注意要用双引号引住,否则Shell会认为你要创建两个文件夹目录。
$ls
picture video doawnload my photos
\end{verbatim}

\hypertarget{manual-pagesman}{%
\paragraph{manual pages(man)}\label{manual-pagesman}}

man是一个程序的说明书,{[}pages{]}处输入你要查询说明的程序名称。

\begin{verbatim}
$man ls
# ls --help也有同样效果
LS(1)                                               User Commands                                               LS(1)

NAME
       ls - list directory contents

SYNOPSIS
       ls [OPTION]... [FILE]...

DESCRIPTION
       List  information  about the FILEs (the current directory by default).  Sort entries alphabetically if none of
       -cftuvSUX nor --sort is specified.

       Mandatory arguments to long options are mandatory for short options too.

       -a, --all
              do not ignore entries starting with .

       -A, --almost-all
              do not list implied . and ..

       --author
              with -l, print the author of each file

       -b, --escape
              print C-style escapes for nongraphic characters

       --block-size=SIZE
              with -l, scale sizes by SIZE when printing them; e.g., '--block-size=M'; see SIZE format below
# 输入 :q 退出查看。
\end{verbatim}

顺带一提,输入\texttt{clear}可以清楚当前终端,使用 control+L 也可以。

\hypertarget{stream}{%
\paragraph{\texorpdfstring{\textbf{stream}}{stream}}\label{stream}}

流。\\
Stream分为 input stream 和 output stream\\
input stream(keyboard)\\
output stream(terminal)

用于重定向流的字符

'\textless' 重定向输入流\\
'\textgreater' 重定向输出流

\begin{verbatim}
$echo hello > hello.txt
# 此时 hello 的返回结果 将会输入到hello.txt这个文本文件内。
# 这里要在相对路径下生效。
\end{verbatim}

如果这时候想要查看hello.txt的内容是否有了hello,需要使用\texttt{cat}指令。

\hypertarget{cat}{%
\paragraph{cat}\label{cat}}

程序验证(cat),用于打印文件内容。

\begin{verbatim}
$cat hello.txt
hello
# 说明重定向输出流成功。
\end{verbatim}

\textbf{cat也支持重定向流。}

\begin{verbatim}
$cat < hello.txt
hello
# 重定向输入流成功。
\end{verbatim}

同时,cat还具有copy功能。

\begin{verbatim}
$cat < hello.txt > hello2.txt
# 此时同样的内容会出现在hello2.txt中。
\end{verbatim}

特殊字符'\textgreater\textgreater',追加(append),向文件末尾继续添加内容(覆写,overwrite)。

\begin{verbatim}
$cat < hello.txt > hello2.txt
$cat < hello.txt >> hello2.txt
$cat hello2.txt
hello
hello
# 第1行,第2行都是hello。后者是追加内容。
\end{verbatim}

\hypertarget{pipe}{%
\paragraph{pipe('\textbar')}\label{pipe}}

管道操作符'\textbar',可以操作(io)流的过程。还可以处理二进制图片或者推流视频文件。

\begin{verbatim}
# 介绍指令 tail -n1 打印文件或者目录的最后(n = 1)行。 可以重定向。
$ls -l | tail -n1
drwxrwxrwx 1 shaymin shaymin 4096 Aug 11 08:35 missing-semester-notes
\end{verbatim}

\hypertarget{tee}{%
\paragraph{tee}\label{tee}}

读取输入,写入到文件并且写入到标准输出流。

\begin{verbatim}
[shaymin@ubuntu2004 /sys]$echo 1060 | sudo tee brightness
1060
# 这里调用权限修改了system目录下的屏幕亮度为1060(尼特)
# 这个操作在WSL(Windows Subsystems for Linux,Windows下的Linux子系统)无法完成。
\end{verbatim}

\textbf{注意:如果不了解,请不要随意修改根目录的文件内容。}

\end{document}
